%--------------------------------------------------------------------------
% Dokumentenklasse
%--------------------------------------------------------------------------

% disable Warning for remreset Package
\RequirePackage{silence}
\WarningFilter{remreset}{The remreset package}

\documentclass[
	pagesize,
	fontsize=12pt,
	paper=a4,
	oneside,
   reqno
]{scrartcl}

%--------------------------------------------------------------------------
% Standartpackete 
%--------------------------------------------------------------------------
\usepackage[ngerman]{babel}               % Deutsch Silbentrennung
\usepackage[T1]{fontenc}                  % Font Type
\usepackage[utf8]{inputenc}               % Font Encoding
\usepackage{lmodern}                      % Latin Modern Font
\usepackage{csquotes}                     % Setzen von Zitaten
\usepackage{xspace}                       % setzten von Leerzeichen nach Abkürzungen
\usepackage{microtype}                    % für glättere Seitenränder
\renewcommand*\familydefault{\sfdefault}  % Serifenlose Schrift
%\renewcommand*\familydefault{\ttdefault} % Schreibmaschinenschrift

%--------------------------------------------------------------------------
% Extra Packages
%--------------------------------------------------------------------------

% Abkürzungspaket
\usepackage{acronym}

% Mathe Pakete
\usepackage{amsmath}
\usepackage{thmtools}
\usepackage{amsfonts}
\usepackage{amssymb}
\usepackage{mathtools}
\usepackage{gensymb}

% Listenumgebungen
\usepackage{listings}
\usepackage{paralist}
\usepackage{enumitem}
\usepackage{adjustbox}

% Demo Text
\usepackage{blindtext}

% Farb-Pakete
\usepackage{xcolor}
\usepackage{fancyvrb}
\usepackage{colortbl}

% Farbedefinitionen
\definecolor{htw}{RGB}{120, 184, 2}
\definecolor{ccW}{RGB}{255,255,255}
\definecolor{ccR}{RGB}{197,14,31}
\definecolor{ccG}{RGB}{113,113,113}
\definecolor{ccL}{RGB}{220,220,220}
\definecolor{ccS}{RGB}{0,0,0}
\definecolor{ccB}{RGB}{68,73,159}
\definecolor{ccD}{RGB}{0,0,80}

% Für erweiterte Tabellen
\usepackage{longtable}
\usepackage{tabularx}
\usepackage{float}
\usepackage{multirow}
\usepackage{makecell}
% \setlength{\tabcolsep}{0.5em}       % for the horizontal padding
% {\renewcommand{\arraystretch}{1.8}  % for the vertical padding
% \usepackage{ragged2e}
% \newcolumntype{R}[1]{>{\RaggedRight}p{#1}}

% Einheitenpaket
\usepackage[exponent-product = \cdot]{siunitx}
\sisetup{locale=DE}

\makeatletter
\renewcommand\@dotsep{5}
\makeatother

% Pakete für Grafiken
\usepackage{graphicx}
\usepackage{wrapfig}
\usepackage{overpic}
\usepackage{epstopdf}
\usepackage{caption}
\usepackage{subcaption}
% \captionsetup[subfigure]{list=true, font=normalsize, labelformat=brace, position=top} %setup für subfigure captions

% Diagramm-/Grafikerstellung
\usepackage{pstricks}
\usepackage{tikz}
\usetikzlibrary{math}
\usepackage{pgfplots}
\pgfplotsset{compat=1.5}
\usetikzlibrary{intersections,positioning,arrows,automata,calc,patterns,shapes.multipart,fit,backgrounds,decorations.pathreplacing}
\usetikzlibrary{decorations,shapes.geometric}
\usetikzlibrary{matrix,calc,angles,positioning,quotes}
% \usepackage{tikz-uml}

\usepackage{pgfkeys}
\usepackage{pgfopts}
\usepackage{ifthen}
\usepackage{xstring}
\usepackage{calc}
\usepackage{pst-plot,pst-bar,pst-node} % Balkendiagramme
\usepackage{capt-of}
\usepackage{incgraph} % Fullscreen Images
\usepackage{pdfpages} % Include external pdf pages

\usepackage{latexsym}
\usepackage{censor}
\usepackage{here}
% \StopCensoring        % Auskommentiert wird der Text entschwaerzt 
% \censor{Oszilloskop}  % Befehl zum einschwärzen
\usepackage{trfsigns}   % Transformation Symbol o---o \laplace and \Laplace
\usepackage{circuitikz}

\usepackage{multido}

% Verlinkungen im Text
\usepackage{url}
\usepackage{hyperref}
\PassOptionsToPackage{hyphens}{url}
\hypersetup{hidelinks}
\urlstyle{same}

%--------------------------------------------------------------------------
% Eigene Befehle
%--------------------------------------------------------------------------

% \renewcommand{\thesection}{\arabic{section}} % Section startet mit 1.0 und nicht mit 0.1

%------------sectioning command-------------------
% The sectioning command one level down the hierarchy from \subsubsection is called \paragraph followed by \subparagraph
% to include this in your table of contents

% for paragraph
\setcounter{tocdepth}{4}
\setcounter{secnumdepth}{4}
% for subparagraph
\setcounter{tocdepth}{5}
\setcounter{secnumdepth}{5}

%------------Zitate-------------------------------
\newcommand*{\zitat}[2]{%
   \normalfont\small
   \begin{quote}
   \glqq#1\grqq \par
   #2
   \end{quote}
   \normalsize
}
\newcommand*{\zitatmitueberschrift}[3]{%
   \normalfont\small
   \begin{quote} #3
   \glqq#1\grqq \par
   #2
   \end{quote}
   \normalsize
}
\newcommand*{\zitext}[2]{%
   \glqq#1\grqq\ %
   [#2]%
}

%-----------Seitendesign--------------------------
\usepackage[width=15.5cm, height=23cm, includeheadfoot]{geometry}
\geometry{paper=a4paper}
% \usepackage[left=6cm,right=1cm,top=1.5cm, bottom=1cm, includeheadfoot]{geometry}
% \newgeometry{oneside}
% \setlength{\voffset}{0cm}
\setlength{\headheight}{1.1\baselineskip} % increase headheight
\setlength{\footheight}{28.99998pt}       % increase foodheight
\setlength{\parindent}{0cm}               % Einrücken nach \newline
\setlength{\footskip}{86pt}               % Move Footer down
% \setlength{\topmargin}{0cm}
% \setlength{\marginparsep}{0.5cm}
% \setlength{\marginparwidth}{1.5cm}
% \setlength{\textwidth}{16cm}
% \setlength{\textheight}{23cm}
% \setlength{\oddsidemargin}{1cm}
% \setlength{\evensidemargin}{2cm}

%----------Kopf & Fußzeile------------------------
% \usepackage[headsepline,footsepline]{scrpage2}
\usepackage[headsepline]{scrlayer-scrpage}
\pagestyle{scrheadings}
\clearpairofpagestyles
\ihead{\headmark}
\automark{section}
\chead{}
\ohead{\includegraphics[scale=0.09]{Bilder/HTWLogoKopfzeile.png} \nocite{HTWklein}}
\ifoot{Aaron Zielstorff\\ 567183}
\cfoot{\pagemark}
\ofoot{M1 Angewandte Mathematik}

%--------------------------------------------------------------------------
% Beginn des Dokuments
%--------------------------------------------------------------------------
\begin{document}

%----------Deckblatt----------------------------- 
\begin{titlepage}
   \pagestyle{empty} % setzt Pagestyle-Befehl

   % HTW Logo
   \begin{flushright}
   \includegraphics[scale=.07]{Bilder/LogoHTWBerlin.png}  \nocite{HTWgross}
   \end{flushright}

   \vspace{1cm}

   % Titel
   \begin{center}
      \Huge{\textbf{Projekt Zeitaufgelöste Photolumineszenz: 
      Angewandte Mathematik (M1)}} \\
   \end{center}

   \vspace{3cm}

   % Name
   \begin{flushleft}
      \begin{tabular}{l l}
         \textbf{Name:}    & Aaron Zielstorff   \\
         \textbf{Mtr.Nr.:} & 567183             \\
      \end{tabular}
   \end{flushleft}

   \vspace{1cm}

   % Daten
   \begin{tabular}{l l}
      \textbf{Fachbereich:}   & FB1                                              \\
      \textbf{Studiengang:}   & M.\xspace Elektrotechnik                         \\
      \textbf{Fachsemester:}  & 1.\xspace FS                                     \\
      \textbf{Fach:}          & M1 Angewandte Mathematik                         \\
      \textbf{Dozent:}        & Prof.\xspace Dr.\xspace A.\xspace Zeiser         \\
      \textbf{Abgabe am:}     & 20.\xspace März 2022                             \\ 
   \end{tabular}
\end{titlepage}
\clearpage

%--------Inhaltsverzeichnis-----------------------
\renewcommand{\contentsname}{Inhaltsverzeichnis}
\tableofcontents
\clearpage

%--------Abbildungsverzeichnis--------------------
\renewcommand{\listfigurename}{Abbildungsverzeichnis}
\renewcommand*{\figurename}{Abb.}
\listoffigures
% \clearpage

%--------Tabellenverzeichnis----------------------
\renewcommand*{\listtablename}{Tabellenverzeichnis}
\renewcommand*{\tablename}{Tab.}
\listoftables
\clearpage


%---------Kapitel/Text----------------------------



%---------Quellen---------------------------------
\newpage
\newcount\Quellennummer
\Quellennummer=1

\renewcommand\refname{Literaturverzeichnis}
\addcontentsline{toc}{section}{Literaturverzeichnis}

\begin{thebibliography}{999}
{\setlength{\emergencystretch}{3cm}%

\bibitem[\the\Quellennummer]{HTWgross}
HTW-Logo auf dem Deckblatt\par
\url{https://de.wikipedia.org/wiki/Datei:Logo_HTW_Berlin.svg} \par
 Stand: 17.08.2018 um 14:49 Uhr

\advance\Quellennummer by 1
 
\bibitem[\the\Quellennummer]{HTWklein}
HTW-Logo in der Kopfzeile\par
\url{http://tonkollektiv-htw.de/} \par
 Stand: 17.08.2018 um 14:53 Uhr

\advance\Quellennummer by 1

\bibitem[\the\Quellennummer]{RannebergScript}
Vorlesungsskript-Ranneberg (2014): LE\_Modul\_M11\_B6-Brücke\par

\advance\Quellennummer by 1

\bibitem[\the\Quellennummer]{RannebergLaboranleitung}
Versuchsanleitung-Ranneberg (2021): Simulationsübung
Leistungselektronik (M8 PCÜ)\par

}
\end{thebibliography}

\end{document}